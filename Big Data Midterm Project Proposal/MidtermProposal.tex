\documentclass{article}
\usepackage[utf8]{inputenc}
\usepackage{hyperref}
\hypersetup{
    colorlinks=true,
    linkcolor=blue,
    filecolor=magenta,      
    urlcolor=cyan,
}
 
\title{Big Data Midterm Project Proposal}
\author{Xingyao Chen and Cassidy Le}
\date{February 2019}

\begin{document}

\maketitle

\section{Overview}
The scarcity of available water resources to provide for the needs of various regions around the world is set to reach an unprecedented and distressing level in the coming decades. This may sound like a topic better reserved for environmental analysts, but to be clear, there are considerable stakes at play. The severity of the situation is set to affect more than just the world economy; water scarcity and its symptoms may very well contribute to increasing sociopolitical and international conflicts, as suffering nations begin to question the distribution of fresh water resources. Furthermore, the increasing effects of climate change are cause for concern regarding the stability and dependability of water resources in the coming decades. Though there exist a deluge of data relevant to the use of and economics surrounding water, the unpredictability of water availability creates a nontrivial statistical task in building an understanding the crisis. 

\section{Materials and Methods}
We plan to use a data set obtained from the The DataOpen Citadel SoCal Datathon; the data schema can be viewed \href{https://drive.google.com/open?id=1fCQeQy01ulQ-UlhwPo4rM6qUNRgwARoE}{here}.
The data set is a collection of water-related demographic and environmental information taken from sources including The Center For Disease Control, U.S. Census, U.S. Department of Agriculture, 
and U.S. Department of the Interior. \\
We will employ regression and classification methods to investigate the associations between population demographics and water quality and availability among various U.S. regions. Upon our analysis, we hope to provide novel insights into the underlying social-economical issues that drive 
this worldwide crisis. 
\end{document}
